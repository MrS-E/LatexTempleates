\documentclass{article}
\usepackage{pgfgantt}

\begin{document}


%
% A fairly complicated example from section 2.9 of the package
% documentation. This reproduces an example from Wikipedia:
% http://en.wikipedia.org/wiki/Gantt_chart
%
\definecolor{barblue}{RGB}{153,204,254}
\definecolor{groupblue}{RGB}{51,102,254}
\definecolor{linkred}{RGB}{165,0,33}
\renewcommand\sfdefault{phv}
\renewcommand\mddefault{mc}
\renewcommand\bfdefault{bc}
\setganttlinklabel{s-s}{START-TO-START}
\setganttlinklabel{f-s}{FINISH-TO-START}
\setganttlinklabel{f-f}{FINISH-TO-FINISH}
\sffamily
\begin{ganttchart}[
    canvas/.append style={fill=none, draw=black!5, line width=.75pt},
    hgrid style/.style={draw=black!5, line width=.75pt},
    vgrid={*1{draw=black!5, line width=.75pt}},
    today=7,
    today rule/.style={
      draw=black!64,
      dash pattern=on 3.5pt off 4.5pt,
      line width=1.5pt
    },
    today label font=\small\bfseries,
    title/.style={draw=none, fill=none},
    title label font=\bfseries\footnotesize,
    title label node/.append style={below=7pt},
    include title in canvas=false,
    bar label font=\mdseries\small\color{black!70},
    bar label node/.append style={left=2cm},
    bar/.append style={draw=none, fill=black!63},
    bar incomplete/.append style={fill=barblue},
    bar progress label font=\mdseries\footnotesize\color{black!70},
    group incomplete/.append style={fill=groupblue},
    group left shift=0,
    group right shift=0,
    group height=.5,
    group peaks tip position=0,
    group label node/.append style={left=.6cm},
    group progress label font=\bfseries\small,
    link/.style={-latex, line width=1.5pt, linkred},
    link label font=\scriptsize\bfseries,
    link label node/.append style={below left=-2pt and 0pt}
  ]{1}{13}
  \gantttitle[
    title label node/.append style={below left=7pt and -3pt}
  ]{WEEKS:\quad1}{1}
  \gantttitlelist{2,...,13}{1} \\
  \ganttgroup[progress=57]{WBS 1 Summary Element 1}{1}{10} \\
  \ganttbar[
    progress=75,
    name=WBS1A
  ]{\textbf{WBS 1.1} Activity A}{1}{8} \\
  \ganttbar[
    progress=67,
    name=WBS1B
  ]{\textbf{WBS 1.2} Activity B}{1}{3} \\
  \ganttbar[
    progress=50,
    name=WBS1C
  ]{\textbf{WBS 1.3} Activity C}{4}{10} \\
  \ganttbar[
    progress=0,
    name=WBS1D
  ]{\textbf{WBS 1.4} Activity D}{4}{10} \\[grid]
  \ganttgroup[progress=0]{WBS 2 Summary Element 2}{4}{10} \\
  \ganttbar[progress=0]{\textbf{WBS 2.1} Activity E}{4}{5} \\
  \ganttbar[progress=0]{\textbf{WBS 2.2} Activity F}{6}{8} \\
  \ganttbar[progress=0]{\textbf{WBS 2.3} Activity G}{9}{10}
  \ganttlink[link type=s-s]{WBS1A}{WBS1B}
  \ganttlink[link type=f-s]{WBS1B}{WBS1C}
  \ganttlink[
    link type=f-f,
    link label node/.append style=left
  ]{WBS1C}{WBS1D}
\end{ganttchart}

%
% A simpler example from the package documentation:
%
\begin{ganttchart}{1}{12}
  \gantttitle{2011}{12} \\
  \gantttitlelist{1,...,12}{1} \\
  \ganttgroup{Group 1}{1}{7} \\
  \ganttbar{Task 1}{1}{2} \\
  \ganttlinkedbar{Task 2}{3}{7} \ganttnewline
  \ganttmilestone{Milestone}{7} \ganttnewline
  \ganttbar{Final Task}{8}{12}
  \ganttlink{elem2}{elem3}
  \ganttlink{elem3}{elem4}
\end{ganttchart}

%
% strongly connected but simple styling
%
\begin{figure}[tbp]
    \begin{center}
    \begin{ganttchart}[
    y unit title=0.4cm,
    y unit chart=0.5cm,
    vgrid,
    hgrid, 
    title label anchor/.style={below=-1.6ex},
    title left shift=.05,
    title right shift=-.05,
    title height=1,
    progress label text={},
    bar height=0.7,
    group right shift=0,
    group top shift=.6,
    group height=.3]{1}{24}
    %labels
    \gantttitle{Week}{24} \\
    \gantttitle{Monday}{4} 
    \gantttitle{Tuesday}{4} 
    \gantttitle{Wednesday}{4} 
    \gantttitle{Thursday}{4} 
    \gantttitle{Friday}{4} 
    \gantttitle{Saturday}{4} \\
    %tasks
    \ganttbar{first task}{1}{2} \\
    \ganttbar{task 2}{3}{8} \\
    \ganttbar{task 3}{9}{10} \\
    \ganttbar{task 4}{11}{15} \\
    \ganttbar[progress=33]{task 5}{20}{22} \\
    \ganttbar{task 6}{18}{19} \\
    \ganttbar{task 7}{16}{18} \\
    \ganttbar[progress=0]{task 8}{21}{24}
    
    %relations 
    \ganttlink{elem0}{elem1} 
    \ganttlink{elem0}{elem3} 
    \ganttlink{elem1}{elem2} 
    \ganttlink{elem3}{elem4} 
    \ganttlink{elem1}{elem5} 
    \ganttlink{elem3}{elem5} 
    \ganttlink{elem2}{elem6} 
    \ganttlink{elem3}{elem6} 
    \ganttlink{elem5}{elem7} 
    \end{ganttchart}
    \end{center}
    \caption{Gantt Chart}

\end{figure}



\newgantttimeslotformat{isotimestamp}{%
  \def\decomposestardate##1-##2-##3 ##4:##5:##6\relax{%
    \def\stardateyear{##1}\def\stardatemonth{##2}\def\stardateday{##3}%
    \def\stardatehour{##4}\def\stardateminute{##5}\def\stardatesecond{##6}%
  }%
  \decomposestardate#1\relax
  \pgfcalendarjuliantodate{\stardateyear-\stardatemonth-\stardateday}{\stardatehour:\stardateminute:\stardatesecond}{#2}%
}

\begin{figure}[ht]
    \centering
    \begin{ganttchart}[
  time slot format=isodate-yearmonth,
  time slot unit=minute,
          x unit=0.5cm, % Adjusts the horizontal unit size for readability
        y unit title=0.7cm, % Vertical space for the task names
        y unit chart=0.7cm % Vertical space for tasks
    ]{2025-03-16}{2025-03-17} % Defines the chart's start and end date
        % Defining tasks
        \gantttask{Task 1}{2025-03-16 00:00}{2025-03-16 04:00} \\
        \gantttask{Task 2}{2025-03-16 06:00}{2025-03-16 12:00} \\
        \gantttask{Task 3}{2025-03-16 14:00}{2025-03-16 20:00}
    \end{ganttchart}
    \caption{Gantt Chart with 2-Hour Units}
\end{figure}

\end{document}